
\thispagestyle{empty}
\pdfbookmark{Preface}{preface}
\label{preface}

\vspace*{20ex}

\begin{flushright}{\Huge Preface}\end{flushright}

\vspace{10ex}
    
\noindent A while ago, I got a mail from a guy down in Germany. It said:

\begin{quote}
Should you be interested, I have converted some of your html tabs to
LaTeX, because I created my own Dylan songbook and wanted it to look
as good as could be.
\end{quote}
  
There were also some pdf files of a couple of songs. To be honest, I
didn't think much about it -- I didn't really see how a LaTeX version
of `some of the tabs' would ever be useful for me. LaTeX --
that wildly complicated markup language which claimed to produce the
most beautiful output, typographically, but at the cost of a steep
learning curve, and a default output which makes everything look like
something from a mathematical journal (because they are all made in
LaTeX).
 
I answered back, politely, I think (I hope). The reply I got in return
mentioned something about making a whole book that one could take to
the local copyshop and get bound.
 
I still wasn't too impressed; I already had such a file -- Adobe
Acrobat could make the whole site into a big PDF file in a whiz, so
why should I consider this anything special?
  
Well, in the end, I did, and I do, with ever greater thrill, joy, and
inspiration, and the project, from which you are now reading this, has
not only turned dylanchords.com into a beautiful book, it has also
become a story of friendship, intellectual stimulation, and
inspiration to learn, which has -- among other things -- led me
(slowly, slowly) to pick up my programming attempts where I left them
in college, after I had made a semi-functional version of Minesweeper
with 8$\times$12 squares in Basic (remember? the programming language which
the school authorities in the eighties thought that everyone needed to
learn, now that the computer age was coming); in the end, it also led
me to finally ditching Windows in favour of Linux, something I should
have done a long time ago.

\subsection*{What Seal can do}

But first things first.
 
This book -- I quickly learned that it was not simply a matter of
stuffing all the tab files into a PDF file and that was that. For
instance, print out some pages from the tab files on the net and try
to play from that, and you will sooner or later -- sooner, I'd guess
-- run into tab systems which are divided in the middle, or verses
which have the chords on one page and the lyrics on the next.
 
Then turn to page~\ldots{}~-- no, wait: \emph{any }page -- in this
book, and you will find everything to be where it should be. Page
breaks break pages, not songs.
 
If you're reading this directly from a PDF file, you will also be able
to use the index and the table of contents as a link page -- quite
handy for a 1500+ pages book, and nothing that my Adobe-generated PDF
dump could ever dream of.
 
And new additions to the site? Changes, revisions? No problem -- they
are incorporated directly the next time you run the program (as long
as you have the updated files, of course).
 
You want just a booklet with the songs from Empire Burlesque instead
of the whole book? Sure, make some small changes to one file, and you
have your `Love Songs from the Eighties' hit parade collection
in your hand.
 
And last but not least: it looks good. There are details which
distinguish a professionally printed page from what you dump from Your
Average Word Processor to your printer. Some of them are considerable
(such as fonts: if Your Average Word Processor is called MS Word, your
font will by default be Times New Roman or Arial -- bad choices,
whichever way you look at it), other are more subtle and will most
likely not be noticed by anyone without a special interest or a
trained eye. Yet, I happen to think that they are important, not only
for the typography freaks who delight in the perfect curve of a
Garamond `n' and who take it as a personal insult if page
margins aren't proportioned according to the Golden Section. But in an
age when most reading is done either from computer screens or from
printouts from browsers or \textsc{MS} Word, where not a thought has been given
to the visual appearance, I see it as the responsibility of anyone who
produces text to make sure they are appealing; to counteract the print
world's equivalent to elevator muzak. It is my firm belief that good
typography will not save the world, but that bad typography ruins it
just a little. Seal counteracts this -- not bad for a piece of
guitar-strummer's helper software, eh?
 
All this and more is done magically by Heinrich K\"uttler's creation,
Seal. Here's what it does, as seen from a layman's perspective: it
takes all the files from whatever version of Dylanchords you have got;
turns it all into LaTeX files, where hyphenations, page breaks, fonts,
layout, and what not is taken care of; generates an index from this;
and outputs it to PDF or postscript. And voil\`a -- you have a book
in your hands, which rivals any chord book you can buy, both in terms
of layout quality, and of usability and versatility.
 
In order for it to work, there was a whole lot that had to be done
with the files on the site. When I started making the site in 1997, I
didn't know much about html, and I used software which knew even
less. Over the years, this had resulted in a jumble of files, some of
which were ok, many of which were horrible, and none of which were
valid files, in any definition of html.
 
But Heiner had put together a script which did away with the worst
outgrowths, and from there, I could clean out the rest. In May 2005
the files were good enough to replace the old ones. Thus, Seal turned
out to have benefits beyond the use of Seal itself.
 
That is just about all I can tell you about it; for the technical
details, ask Heiner. What I know is: it works!

\subsection*{What you can do}
 
What you can do? Well, you can do anything you can with any other pdf
file, such as: print it out or send it to your friends, but that's not
what I was going to say. The contents is released under the Creative
Commons (CC) licence. This means that you are free -- and encouraged:

\begin{itemize}
  \item to copy, distribute, display, and perform the work
  \item to make derivative works
\end{itemize}

as long as you:
\begin{itemize}
  \item attribute the work in the manner specified by the author or
    licensor, and
  \item don't use it for commercial purposes.
  \item If you alter, transform, or build upon this work, you may
    distribute the resulting work only under a license identical to
    this one.
\end{itemize}
 
In other words: just like the dylanchords site, the contents is
distributed freely, available for anyone who wants to play some good
music, and -- hopefully -- learn something along the way. The
conditions are that the attribution is retained, that you don't make
any money from it (I don't count the free beer you get, playing from
it in your local pub), and that if you use it in a ``derivative
work'', e.g. include it in teaching material or make your own book,
this new work should also be made publicly available under the same
conditions.
 
The intention is to make sure the material is and will remain freely
available, but without abandoning all control. That is why the CC
licence is also labeled ``Some Rights Reserved''{}. It is not a
complete ``copyleft''{}.
 
It goes without saying that this applies only to the parts of the
contents which is in some way or another my ``intellectual property''
-- the introductions and instructions, of course, but even the chord
charts fall under this category, even though Bob Dylan, as the
copyright holder of the original work, has the right to decide about
their publication. The same, naturally, goes for the lyrics (where my
contribution is more modest: correcting some errors in the published
versions, and, probably, adding some new ones).
 
I've been hesitant to put a CC banner on the site before because of
this -- I wouldn't want to postulate a publishing licence for
Dylan's work -- but I now feel more confident and justified, both
because the context is different, and because I now know more about
the legal issues involved.
 
For me, this is a way of responding to the statement ``Everybody must
give something back for something they get''. Working this closely
with Dylan's music over the years has given me tremendously much: a
deeper insight in one of the most remarkable musicians in modern
Western culture; a peek into the musical universe populated by the
likes of Dock Boggs, Woody Guthrie, heck, even Hank Williams, which I
would otherwise never have touched but which has been opened up with
Dylan as a guide; some great friends; some html skills; and an
opportunity to tune my ear (and my guitar). This is my way of paying
back.
