% Taken from roadmaps.htm on Sun May 21 11:46:11 CEST 2006

\phantomsection
\thispagestyle{empty}
\pdfbookmark{Roadmaps}{roadmaps}
\label{roadmaps}

\vspace*{20ex}

\begin{articlelayout}
\begin{flushright}{\Huge Roadmaps for the Soul}\end{flushright}

\vspace{10ex}

\subsection*{General principles}

Just a few words about the principles that (mostly) have been followed
in making the tabs on this site.

First of all: this is a guitar site, not a ``{}chord''{} site. The
ideal ``{}readers''{} I have in mind are the average (or
average-to-good-to-very-good) guitar players, playing for their own
enjoyment (let's not talk about the neighbours -- just love
them). That means on the one hand that I transcribe the songs into
what is convenient to play on a guitar, not necessarily into what is
actually sounding -- in other words: I use the capo, just like Bob Dylan
himself. Just because a song happens to be played in the key of Eb
major, doesn't mean that it has to be tabbed in that key, when it is
actually played in ``{}C major''{} with a capo on the third fret.

2. A corollary of this is that my aim is not just giving the chords of
a song, but also to figure out as exactly as possible what is being
played in the version up for study.

3. On the other hand there is the problem of the ``{}piano songs''{}
and the ``{}full band songs''{}. Whereas Dylan prefers (or preferred;
things have changed) the keys of C major and G major on the guitar, he
delights in odd, awkward keys with lots of black keys, especially
C$\sharp$ major (or D$\flat$ major), when he's at the piano. In these
cases it is of course impossible to reproduce exactly what is being
played. I still use a capo (of course). In the ``{}full band songs''{}
-- especially in later years -- Dylan usually just plays the chords,
in any position, usually with barre chords somewhere up on the neck.

This means that there are three main types of tabs/chord-files on this
site, corresponding with three kinds of arrangements: (1) The solo
acoustic songs, or songs where the guitar work is of some
prominence. Here exactness is a goal. (2) Then there is the big group
of songs where no particular instrument is prominent, least of all
Dylan's guitar. Here the aim of the tabs is to present an
approximation of what is going on in the song, rather than figuring
out exactly what Dylan is playing (which in these cases is quite
uninteresting, actually). (3) The piano songs constitute a
sub-division of this group.

Many of the tabs are of rare live songs. I presume that anyone who
would want to use the tabs already knows what the songs sound
like. Therefore, more specific performance indications, apart from the
fingerings, are considered unnecessary.

\subsection*{Chords and Chord Names}

\paragraph*{Chords and scales -- a little theory and some terms.} A
chord is a selection of tones which are perceived as a unity and not
just as several notes sounding at the same time. It gets its special
character to a large extent thanks to the place the tones have in the
\emph{tonal system}, which, slightly simplified, means the hierachical
system of relationships between the relevant tones of a song or a
style in general. In C major, the tone \textbf{c} is more central than
\textbf{a}, which again is more central than \textbf{f sharp}.

It is customary to arrange the available tones in a \emph{scale}, a
``{}ladder''{}, and to refer to them according to their position in
the scale. The keynote is called ``{}prime''{}, the tone above it
``{}second''{}, the next ``{}third''{}, etc. Thus, the tones in a C
major scale would be called:
\begin{pre}\begin{adjustwidth}{2em}{0em}
{-}{-}{-}{-}{-}{-}{-}{-}{-}{-}{-}{-}{-}{-}{-}{-}{-}{-}{-}{-}{-}{-}{-}{-}{-}{-}{-}{-}{-}{-}{-}{-}{-}{-}{-}{-}{-}{-}{-}{-}{-}{-}{-}{-}{-}{-}{-}{-}{-}{-}{-}{-}{-}{-}\\*
{-}{-}{-}{-}{-}{-}{-}{-}{-}{-}{-}{-}{-}{-}{-}{-}{-}{-}{-}{-}{-}{-}{-}{-}{-}{-}{-}{-}{-}{-}{-}{-}{-}{-}{-}{-}{-}{-}{-}{-}{-}{-}0{-}{-}{-}{-}{-}{-}1{-}{-}{-}{-}\\*
{-}{-}{-}{-}{-}{-}{-}{-}{-}{-}{-}{-}{-}{-}{-}{-}{-}{-}{-}{-}{-}{-}{-}{-}{-}{-}{-}{-}-0{-}{-}{-}{-}-2{-}{-}{-}{-}{-}{-}{-}{-}{-}{-}{-}{-}{-}{-}{-}{-}{-}{-}\\*
{-}{-}{-}{-}{-}{-}{-}{-}-0{-}{-}{-}{-}{-}{-}2{-}{-}{-}{-}-3{-}{-}{-}{-}{-}{-}{-}{-}{-}{-}{-}{-}{-}{-}{-}{-}{-}{-}{-}{-}{-}{-}{-}{-}{-}{-}{-}{-}{-}{-}-\\*
{-}{-}3{-}{-}{-}{-}{-}{-}{-}{-}{-}{-}{-}{-}{-}{-}{-}{-}{-}{-}{-}{-}{-}{-}{-}{-}{-}{-}{-}{-}{-}{-}{-}{-}{-}{-}{-}{-}{-}{-}{-}{-}{-}{-}{-}{-}{-}{-}{-}{-}{-}{-}-\\*
{-}{-}{-}{-}{-}{-}{-}{-}{-}{-}{-}{-}{-}{-}{-}{-}{-}{-}{-}{-}{-}{-}{-}{-}{-}{-}{-}{-}{-}{-}{-}{-}{-}{-}{-}{-}{-}{-}{-}{-}{-}{-}{-}{-}{-}{-}{-}{-}{-}{-}{-}{-}{-}{-}\\*
~~c~~~~~~d~~~~~~e~~~~~f~~~~~~g~~~~~a~~~~~~b~~~~~~c'\\*
~prime~second~third~fourth~fifth~sixth~seventh~octave
\end{adjustwidth}\end{pre}

The most weighty tones in the scale are the \emph{prime} (or
\emph{unison}), the \emph{fifth}, and the \emph{third} (in C major:
\textbf{c}, \textbf{g} and \textbf{e}). Then follow the remaining
tones in the main scale (\textbf{d, f, a}, and \textbf{b}), and lastly
the tones that are extraneous to the scale: the semi-tones (\textbf{f
sharp, e flat}, etc.)

I mention the fifth before the third, but they are important in
different ways. The fifth is stable, a loyal companion to the prime,
always there, not without its conflicts, but they are always resolved,
and always in favour of the prime -- somewhat like a good old (or bad
old, depending on the perspective) patriarchal marriage. In fact, one
might consider all music within the western musical tradition (until
the late nineteenth century in the art-music tradition, and until this
day in the popular traditions) as nothing more than a play with the
balance between these two scale steps.

As I said, the fifth is always there. When you strike a string, it
will vibrate in many different ways. The whole string will swing and
produce the loudest tone. But all the possible equal divisions of the
string will also swing, and produce \emph{overtones}. The difference
in sound between different instruments is caused by different
constellations of overtones -- which are strong and which are not. The
division of the string in two (at the twelfth fret) will sound an
octave higher, i.e. with a tone of the same pitch class, which will
strengthen the basic tone further. But the division in three, at the
seventh fret, will produce the fifth. (Exercise: strike a bass string
while touching it at the seventh fret, but without pressing it
down. Then play the open string, and you should be able to hear the
fifth \emph{in} the full tone of the open string.) Thus, If you play a
\textbf{c}, you will also hear a \textbf{g}.

The third is a different matter. Where the fifth gives support and
reenforcement, the third adds character. It is unstable, at times
nervously shimmering, other times over-earthly sonorous. It can not be
defined as easily as the fifth. It lies two divisions above the fifth
in the series of overtones (you can isolate it on a string by touching it,
without pressing the finger down, at the fourth fret; this will divide the string
in five) and therefore sounds less strongly than the
fifth. Furthermore, it exhibits a peculiarity of the tonal system
which has plagued theoreticians since the days of Pythagoras: if one
stacks four fifths on top of each other -- \textbf{c-g, g-d, d-a} and
\textbf{a-e} -- one might think that one gets to the same
\textbf{e} as when one divides a string in five, but one doesn't --
one gets to a tone that lies considerably higher (\emph{c.} a quarter
of a semi-tone, which is quite a lot). This is not really a problem,
but an opportunity: tension is the mother of all development, and the
third is as tense as it gets.

The most significant difference between the fifth and the third,
though, is that, whereas there is only one fifth, there are two
possible places for the third. Both \textbf{c-e} and \textbf{c-e flat}
are thirds, but one is \emph{major}, the other \emph{minor}. The third
is the interval which decides the most fundamental character of a
chord: whether it is major or minor. \textbf{C-e-g} is a C major
chord, \textbf{c-e flat-g}  is C minor. The same distinction can be
drawn on the second, sixth, and seventh steps, whereas the prime, the
fifth, and the fourth can only be (violently!) augmented or
diminished.

\paragraph{For ``losers, cheaters, six-string abusers'' (ain't we
all\ldots).}  The tabs present what is being played, by trained and
proficient musicians (yes, I'm counting in Dylan). On the one hand
Dylan is an ideal artist for a beginner, since he always uses quite
simple and logical chord shapes, and licks and tricks that let him get
maximum effect from minimum effort. Still, the beginner may run into
problems, with strange chord names, barre chords etc. Here's just a
few cheats.

(1) All chords, basically, go back to the three fundamental chords in
a key (in C: C, G and F) and their minor relatives (Am, Em, Dm). Most
frequent are the variations to the dominant chord, i.e. the chord on
the fifth step above the key note (G in this example), where the
variations are different ways of creating and sustaining tension
before the return to the key note. That means that ``{}strange''{}
chord names can often be replaced by the simple chord without all the
fuzz behind it (G$\flat$+, E7-10, Dm7-5, Cadd9 become G$\flat$, E, Dm,
C). This does not happen without loss: the ``{}fuzz''{} is there for
some reason (e.g. the E7-10 is the quintessential blues chord, which
is minor and major at the same time; it is an E chord, but the plain E
does not get the same effect), but functionally the plain chord will
usually do the job adequately.

(2) Chords can be replaced with their relatives. When I was nine,
before I had the finger strength to play barre chords, I discovered
that I could replace most F chords with Dm or Am -- one of those would
usually work. Now I know that the reason why it works is that they
both share two out of three chord tones with F, which often is
enough. I don't recommend this method, however (unless you're
nine). It \emph{is} cheating, and the only person you're fooling, in
the long run, is yourself.

(3) Some songs are consistently noted with chords like A$\flat$, Eb,
B$\flat$ etc. That is because they are played with those chords, as
barre chords, and in those cases I've seen no reason to introduce a
capo. The easiest way to avoid those barre chords, is to drop all the
bs, and play E, B, A instead. This only works if all chords have a b
attached to them, though. Other chords you'll have to transpose based
on the thorough knowledge of the outline of the fretboard that you've
gained, e.g. from the figure below.

\paragraph*{Chord short-hand.} I usually present the chords used in
the song, unless it should be obvious (someone who doesn't know how to
play a C major chord probably doesn't have anything to do in here
anyway\ldots{}). Chords are presented with one number for each string,
beginning with the lowest (6th) string. An open string is 0, a finger
on the 3rd fret is 3 etc. An unused string is marked by `{}x'{}, and
strings that are disregarded are marked `{}-'{}. Thus C major looks
like this: x32010, and the recurring fill in ``{}Blood in my Eyes''{}
like this: x32010 -53{-}-{-} -64{-}-{-} -75{-}-{-}.

\paragraph*{Bass notes.} I prefer to write the chords with the key
note as the lowest bass note in the chord. Thus, even though C major
can be played 032010 or 332010 (and often should be), either of the
tones on the 6th string will disturb the ``{}C-majority''{} of the
chord, and is better left out, unless they are explicitly wanted,
e.g. in a running bass progression.

A chord with a bass note other than the keynote is indicated with a
slash between the chord name and the bass note: C/g is a C major chord
with G as the lowest tone: 332010.

The slash and the bass note can be used alone to indicate a bass
progression against a sustained chord: C /b /a /g (x32010 -- x22010 -- x02010 -- 332010).

I usually use lower-case letters for these bass tones, because it
looks less ugly.

\paragraph*{Chord names.} In general I use the following system
(exemplified with C chords throughout): Major chords: C. Minor chords:
Cm. The following table explains the additional symbols and chord
types. All the examples are variants of C. Third = the third note of
the scale from the key note, fifth = the fifth note of the scale,
etc. Since there are only seven different steps in the scale, the
second is the same as the ninth, the fourth is the same as the
eleventh etc. In chord names one will usually use the higher of these,
except where the basic triad is altered;~ e.g. C9 and not C2 (but
Csus4 and Cm7-5). The reason for this is that all chords are considered as stacks of thirds over the keynote. The simplest chords consist of the first two thirds, so C = c-e-g. All the more ``{}sophisticated''{}
chords are considered as extensions of the basic chord with selections
from the stack of thirds above it: c-e-g continues B$\flat$-d-f-a,
which are the 7th, the 9th, the 11th and the 13th. The convention is
that a single number (e.g. 11) indicates how far up the stack goes,
denoting the \emph{last member} of the
stack, not just a single tone: C11 consists of the all
the tones in the stack, up to the eleventh, thus: c-e-g-b$\flat$-d-f.

%\newcommand{\chordexpl}[3]{% Symbol / Name / Example; then Meaning
%  \vspace{3ex}
%  \begin{tabular}{cl}
%  {\mbox{\Huge #1}}&
%  \begin{tabular}{l}
%  \textit{#2}\tabularnewline
%  #3\tabularnewline
%  \tabularnewline
%  \end{tabular}\tabularnewline
%  \end{tabular}%
%}

%% From memman.tex, copied blind.
% Symbol / Name / Example; then Meaning
\newcommand{\chordexpl}[3]{\vspace{1ex}{\noindent
    \setbox0\hbox{\Huge #1~~}%
    \count0=\ht0                   % height of versal
    \count1=\baselineskip          % baselineskip
    \divide\count0 by \count1      % versal height/baselineskip
    \dimen1 = \count0\baselineskip % distance to drop versal
    \advance\count0 by 1\relax     % no of indented lines
    \dimen0=\wd0                   % width of versal
    \global\hangindent\dimen0      % set indentation distance
    \global\hangafter-\count0      % set no of indented lines
    \hskip-\dimen0\setbox0\hbox to\dimen0{\raise-\dimen1\box0\hss}%
    \dp0=0in\ht0=0in\box0}%
    \textit{#2}\\#3\par\nobreak%
}

\begin{adjustwidth}{5em}{0em}

\chordexpl{C7}{\textnormal{(}minor\textnormal{)} seventh}{x32310}
\noindent The minor seventh is added to the root chord. Note that
``minor'' here refers to the tone on the seventh step (which can be both
major and minor: B$\flat$ and B), not to the chord itself i.e., it is a chord
with a minor seventh, not a minor chord with a seventh -- cf. the ``m7'' chord
below. Note also that ``7'' alone always refers to the \emph{minor} seventh. If the
major seventh is used, it has to be specified with ``maj7''. The ``minor'' is usually left out of the name.

\chordexpl{Cmaj7}{major seventh}{x32000}
\noindent The major seventh is added to the root
chord. Whereas the seventh chord usually has a dominant function,
i.e. is used to lead back to the chord five steps lower (C7$\to$F), the
major seventh is rather a colouring of the chord, without this
``driving'' effect.

\chordexpl{Cm7}{---}{x35343}
\noindent The (minor) seventh is added to the minor
chord. Cf. the ``7'' chord above.

\chordexpl{Cm7-5}{---}{x34340}
\noindent The fifth of the m7 chord is lowered by a semitone.

\chordexpl{C9}{ninth}{x32330}
\noindent The ninth \emph{and} the seventh are added to the root
chord.

\chordexpl{C+}{augmented}{x32110} % was "+ (aug)" in table
\noindent The fifth is raised by a semitone (half
step=one fret)

\chordexpl{Co}{diminished}{x34242} % was "o (dim)" in table
\noindent A stack of minor thirds. Since all the intervals in the chord are
equal, any of the tones can function as root. Thus:
Co=E$\flat$o=F$\sharp$o=Ao. Hence, there only exists three different dim chords.

\chordexpl{C11}{eleventh}{x33333}
\noindent The seventh, ninth and eleventh are added to the
root chord. Since these three tones make up the chord on the tone one
step below the root (for C: B$\flat$), this chord usually functions as a
conflation of these two chords.

\chordexpl{C6}{sixth}{x35555}
\noindent The sixth is added to the root chord.

\chordexpl{Csus4}{suspended fourth}{x33010}
\noindent The third is temporarily ``suspended'': raised to the fourth,
and left there hanging in wait for a resolution back to the root chord. Thus,
in a true sus4 chord, the third is not included. If that were the case,
the chord would be called add11 or add4. Note: the chord Cmsus4 is identical to Csus4 and might for this reason be considered redundant. But whereas the sounds are identical, the functions are not: the name indicates that this is \emph{in fact} a Cm chord, it's just a little indisposed at the moment.

\chordexpl{Csus2}{---}{x30010}
\noindent Same as the previous, only that the third ``hangs'' below, on
the second.

\chordexpl{C7-10}{---}{x3234x}
\noindent The blues chord \emph{par exellence}. Since it contains both
the major and the minor third, the chord corresponds to the ambiguity
of the third step in the blues scale.  This chord is usually called
7+9 (or 7\#9), but since the extra tone really functions as a low third
(=tenth) and not a raised second, I prefer the name 7-10 (the raised
ninth and the lowered tenth are of course the same tone on the guitar,
although they are functionally different. Subtleties, subtleties!).

\chordexpl{Cadd\textit{x}}{---}{}
\noindent Any added tone that does not fall within the stack of
thirds upon which the rest of the system is based. A special case is chords containing tones which do belong to the stack, but not all the members below it. E.g., in the chord c-e-g-d (x32030), the d is the ninth, but since the seventh (b$\flat$) is missing, the chord must be called Cadd9 and not C9.
Note the difference between Cadd9, which is a full C with an added d, and Csus2, which is a plain C major chord where the third is temporarily suspended downwards to d.

\chordexpl{C-\textit{x}/C+\textit{x}}{---}{}
\noindent Lowers/raises a scale step by a semitone (one
fret). E.g. Cm7-5 and C7+13. Note: ``+'' does not mean that the 13th
is added, but that it is raised.

\chordexpl{C5}{``Power chord''}{x355xx}
\noindent A chord containing only the prime (the root) and the
fifth. In other words: a chord without the third. Since the third is
the tone that defines whether a chord is major or minor, the ``power
chord'' is neutral in this respect.

\chordexpl{C(iii)}{---}{x35553}
\noindent A chord in the third position, i.e. fingered so that
it begins on the third fret. Thus, the quality of the chord is not
changed, only its sonority. (I have not been quite consistent
concerning this notation, mostly due to the fact that the parentheses
are space-consuming.)

\end{adjustwidth}

\bigskip

\noindent These additions to the chord names can be combined in just about any way you like:
Cmmaj9, Cadd-9add13, Cm7-5,  etc. Heck, you could even write Cmaj7add7 (x32303)

I usually also prefer simple names to ``exact'' names. A chord
like 3x3211 should perhaps (but not necessarily) be called G11, but I
prefer to call it F/g, since that more immediately says what is to be
played (and because it retains the ambiguity inherent in the chord,
between the subdominant and the dominant, which is so central to
Dylan's tonal language). See Blood in my Eyes for a more extreme
case. (I'm beginning to change my mind on this, though. In the more
recent tabs, you'll see G11 more often than F/g).

Approximated chord names are written like "G6" (x33000) or
F$\sharp$m7' (202200) for brevity.

Any chord can be fingered in many different ways. ``C'' does not
``mean'' x32010 -- that is just the simplest and usually most
convenient way to finger it. To get from~ chord name to a chord, you
have to know where the tones are positioned on the fretboard. The
tones are distributed on the strings as follows (e' is the lightest
string, E is the darkest):

\begin{pre}
\begin{adjustwidth}{2em}{0em}
e'||-f'-|-f\#'|-g'-|-g\#'|-a'|-\\*
b~||-c'-|-c\#'|-d'-|-d\#'|-e'|-\\*
g~||-g\#-|-a-{}-|-bb-|-b-{}-|-c'|-~etc.\\*
d~||-d\#-|-e-{}-|-f-{}-|-f\#-|-g-|-\\*
A~||-Bb-|-B-{}-|-c-{}-|-c\#-|-d-|-\\*
E~||-F-{}-|-F\#-|-G-{}-|-G\#-|-A-|-
\end{adjustwidth}
\end{pre}

To find a chord like Am/f$\sharp$ (the most important chord in Trying
to Get to Heaven), start with the basic chord (Am) and search out the
bass tone (f$\sharp$) on one of the darkest strings, where it can be
played. In this case there are two possibilities: on the 4th string:

\begin{pre}
\begin{adjustwidth}{2em}{0em}
e'||-f'-|-f\#'|-g'-|-g\#'|-a'|\\*
b~||-\textcolor{red}{c'}-|-c\#'|-d'-|-d\#'|-e'|\\*
g~||-g\#-|-\textcolor{red}{a}-{}-|-bb-|-b-{}-|-c'|\\*
d~||-d\#-|\textcolor{red}{(e)}-|-f-{}-|-\textcolor{red}{f\#}-|-g-|\\*
A~||-Bb-|-B-{}-|-c-{}-|-c\#-|-d-|\\*
E~||-F-{}-|-F\#-|-G-{}-|-G\#-|-A-|
\end{adjustwidth}
\end{pre}

or on the 6th: % 5th string f# (4th fret) was also red. Error?
%fixed. eo
\begin{pre}
\begin{adjustwidth}{2em}{0em}
e'||-f'-|-f\#'|-g'-|-g\#'|-a'|\\*
b~||-\textcolor{red}{c'}-|-c\#'|-d'-|-d\#'|-e'|\\*
g~||-g\#-|-\textcolor{red}{a}-{}-|-bb-|-b-{}-|-c'|\\*
d~||-d\#-|-\textcolor{red}{e}-{}-|-f-{}-|-f\#-|-g-|\\*
A~||-Bb-|-B-{}-|-c-{}-|-c\#-|-d-|\\*
E~||-F-{}-|-\textcolor{red}{F\#}-|-G-{}-|-G\#-|-A-|
\end{adjustwidth}
\end{pre}

The second fingering is probably the best one, since it produces a
fuller chord, and since you can use all the strings -- unless the higher sound is precisely what you
want, in which case the first fingering is
better. In that case, xx4555 is a third alternative. It even has the
advantage of having the key note (A) on the highest string, thus
emphasising it.

In the same way we can find the fingering for the chord Bm7-5. First
find the tones: Bm = b, d, f$\sharp$. Add the 7th (a) and lower the
5th (f$\sharp$ $\to$ f), and we have the tones b, d, f and a.

\begin{pre}
\begin{adjustwidth}{2em}{0em}
e'||-\textcolor{red}{f'}-|-f\#'|-g'-|-g\#'|-\textcolor{red}{a'}|-\\*
\textcolor{red}{b}~||-c'-|-c\#'|-d'-|-d\#'|-e'|-\\*
g~||-g\#-|-\textcolor{red}{a}-{}-|-bb-|-b-{}-|-c'|-~etc.\\*
\textcolor{red}{d}~||-d\#-|-e-{}-|-f-{}-|-f\#-|-g-|-\\*
A~||-Bb-|-\textcolor{red}{B}-{}-|-c-{}-|-c\#-|-d-|-\\*
E~||-\textcolor{red}{F}-{}-|-F\#-|-G-{}-|-G\#-|-\textcolor{red}{A}-|-
\end{adjustwidth}
\end{pre}

We probably want the key note (b) in the bass, which in practice
leaves us with the alternatives x2323x or x2x231. A third possibility is xx(3)435. (Note:
Am/f$\sharp$ and Bm7-5 are actually chords of the same
type. Am/f$\sharp$ is the same chord as F$\sharp$m7-5. Try it!)

%For a more comprehensive guide to guitar chords, see the ONLINE
%GUITAR CHORD DICTIONARY, or the other resources at Guitar Notes.

\subsection*{Reading Tab}

The principles I've followed in the tabs have varied a little over the
years, but the following points apply, as a rule, to all files:

The rhythm is indicated above the tab, with dots for each beat and :
for the heavier beats:

\begin{pre}:~~~.~~~.~~~.~~~:~~~.~~~.~~~.\end{pre}

In the cases where an even finer subdivision is needed, a comma is
used:

\begin{pre}:~~.~~,~~.~~,~~.~~,~~.\end{pre}

As far as possible I let the tabs be a graphical image of the rhythms,
so that two spaces are of equal duration anywhere in the tab. That way
one can easily differentiate between the triple time feel of this
example

\begin{pre}
\begin{adjustwidth}{2em}{0em}
~~:~~~~~.~~~~~.~~~~~.\\*
|{-}0{-}{-}{-}0{-}{-}{-}{-}{-}{-}{-}{-}{-}{-}{-}{-}{-}{-}{-}{-}{-}{-}{-}{-}|\\*
|{-}1{-}{-}{-}1(0{-}{-}{-}0{-}0{-}{-}{-}0{-}0{-}{-}{-}0){-}|\\*
|{-}0{-}{-}{-}0(0{-}{-}{-}0{-}0{-}{-}{-}0{-}0{-}{-}{-}0){-}|\\*
|{-}2{-}{-}{-}2{-}3{-}{-}{-}3{-}4{-}{-}{-}4{-}5{-}{-}{-}5{-}{-}|\\*
|{-}3{-}{-}{-}3{-}5{-}{-}{-}5{-}6{-}{-}{-}6{-}7{-}{-}{-}7{-}{-}|\\*
|{-}{-}{-}{-}{-}{-}{-}{-}{-}{-}{-}{-}{-}{-}{-}{-}{-}{-}{-}{-}{-}{-}{-}{-}{-}{-}|
\end{adjustwidth}
\end{pre}

and the square rhythms of this (both from Blood in My Eyes):

\begin{pre}
\begin{adjustwidth}{2em}{0em}
~~:~~~~~.~~~~~.~~~~~.\\*
|{-}0{-}{-}0{-}{-}{-}{-}{-}{-}{-}{-}{-}{-}{-}{-}{-}{-}{-}{-}{-}{-}{-}{-}{-}|\\*
|*1{-}{-}1{-}(0{-}{-}0{-}{-}0{-}{-}0{-}{-}0{-}{-}0){-}*|\\*
|{-}0{-}{-}0{-}(0{-}{-}0{-}{-}0{-}{-}0{-}{-}0{-}{-}0){-}{-}|\\*
|{-}2{-}{-}2{-}{-}3{-}{-}3{-}{-}4{-}{-}4{-}{-}5{-}{-}5{-}{-}{-}|\\*
|*3{-}{-}3{-}{-}5{-}{-}5{-}{-}6{-}{-}6{-}{-}7{-}{-}7{-}{-}*|\\*
|{-}{-}{-}{-}{-}{-}{-}{-}{-}{-}{-}{-}{-}{-}{-}{-}{-}{-}{-}{-}{-}{-}{-}{-}{-}{-}|
\end{adjustwidth}
\end{pre}

Repeats are indicated as in the previous example, or as written-out instructions
(``x3'')

Sometimes I've indicated rhythms also in the ``{}chords''{} part of
the files. Then the bars are indicated, and the main pulse within each
bar. I'm sorry to say that I haven't followed any consistent system to
denote subdivisions of the beat, but I've often joined such chords
together with a hyphen:

\begin{alltt}| A . . . | D . A . |E A/e-E . . |\end{alltt}

The last bar might be tabbed:

\begin{pre}
\begin{adjustwidth}{2em}{0em}
~~:~~~.~~~.~~~.\\*
|{-}0{-}{-}{-}0{-}0{-}0{-}{-}{-}{-}{-}{-}{-}|\\*
|{-}0{-}{-}{-}2{-}0{-}0{-}{-}{-}{-}{-}{-}{-}|\\*
|{-}1{-}{-}{-}2{-}1{-}1{-}{-}{-}{-}{-}{-}{-}|\\*
|{-}2{-}{-}{-}2{-}2{-}2{-}{-}{-}{-}{-}{-}{-}|\\*
|{-}2{-}{-}{-}{-}{-}{-}{-}2{-}{-}{-}{-}{-}{-}{-}|\\*
|{-}0{-}{-}{-}0{-}{-}{-}0{-}{-}{-}{-}{-}{-}{-}|
\end{adjustwidth}
\end{pre}

\noindent\textbf{Special signs:}
\begin{alltt}
{\small
\textsl{Sign  Meaning     Usage}
p     pull-off    2p0
h     hammer-on   0h2 (or h2 if obvious or too fast
                       to be significant)
/     slide up
\bs     slide down
b     bend        3b5 = finger the string at the third
                        fret, and bend it up until it
                        sounds as if it was fingered
                        at the fifth fret.
r     release     release the bended string
                  to normal position.}
\end{alltt}

\subsection*{Open/alternate tunings}

For some of the songs, Dylan uses alternate or open tunings. An \emph{open}
tuning is a tuning where all the strings are tuned to a chord, whereas
\emph{alternate} tunings are altered in some other way.

\subsubsection*{Open tunings}

There were tuned instruments before the guitar's ancestors. They were
usually tuned in open fifths, usually with drone strings and one or
two melody strings. The baroque lute was tuned to an open d minor
chord (with additional bass strings). The main advantage of the
fourths/third tuning that we use, is the possibility of creating
simple fingering patterns for \emph{many different} chords in the same
position.

An obvious consequence of open tunings is that playing is more limited
to the key to which the open strings are tuned. The benefits are quite
simple chord shapes, at least for the basic chords, which makes it
easier to do fancy things on top of those chords; furthermore, unless
one produces the other chords by simply putting a barre across all the
strings, there will usually be open, sounding strings in all chords,
thus giving a handy set of fancy-chords-with-very-long-names.

The most common open tunings (and the only ones encountered in Dylan's
production) are open D, open E, open G, and (in two songs) open A.

\paragraph*{Open D and E}
%\marginpar{D A d f$\sharp$ a d'\\E B e g$\sharp$ b e'}
are basically the same tuning, only one tone
apart. Open E gives a brighter sound, which may be preferable, but it
has the nasty side-effect of also producing the sharp sound of a
broken string more often, and of putting extra strain on the neck of
the guitar, so it is recommended to tune to open D and use a capo on
the 2nd fret. Open D/E is encountered in a number of the songs on
\emph{Freewheelin'}, and the entire \emph{Blood on the Tracks} was
originally recorded in this tuning. For a more thorough presentation
of Dylan's use of the open D/E tuning, I refer to my introductory
notes on Blood on the Tracks.

\noindent Open D: D A d f$\sharp$ a d'\\
Open E: E B e g$\sharp$ b e'

\emph{Songs.} Highway 51 --- In My Time Of Dying --- Roll On John ---
Two Trains Running --- I shall be Free --- Corrina
Corrina --- Oxford Town --- Gypsy Lou --- Tomorrow is a Long Time ---
Standing On The Highway --- Rambling Gambling Willie --- Walkin' Down the
Line --- Whatcha Gonna Do? --- Ballad For A Friend --- \textbf{Blood On The
Tracks} (\textit{all the songs})

\paragraph*{Open G} is the most common slide guitar tuning, popular among
delta blues players. Since Dylan was an old delta blues player himself
early in his carreer, you'll find a few songs in this tuning. The only
song on this site, though, is I Was Young When I Left Home.

\noindent Open G: D G d g~b d'

\paragraph*{Open A.} Two songs uses a completely different tuning: the
Freewheelin' outtake Wichita and One too many mornings.

\begin{tabbing}
Open A: \= E A c$\sharp$ e a e' (Wichita Blues) or\\
        \> E A c$\sharp$ e a c$\sharp$' (One Too Many Mornings)\\
\end{tabbing}

\subsubsection*{Alternate tunings}

Again, there are really only three different tunings to keep track of
in Dylan's catalogue: drop D, drop C and double drop D (to my
knowledge he's never played ``{}drop dead''{}). They all involve the
6th and deepest string: in drop D, the 6th string is tuned one step
down, and in drop C, two steps. In double drop D both the 1st and the
6th strings are tuned down to D.

\begin{tabbing}
Standard tuning: \= E A d g b e'\\
Drop D:          \> D A d g b e'\\
Drop C:          \> C A d g b e'\\
Double drop D:   \> D A d g b d'\\
\end{tabbing}

All these tunings have their own distinct sets of chords, always
centering around the deepest bass tone. An example is the chord G. In
drop D tuning, the central chord is D (000232). Thus the natural way
to finger G is 020033. In drop C, on the other hand, the central chord
is C (032010), and the most comfortable version of G is 220001. This
is a G7 chord, and this is consequently the only tuning in which Dylan
consistently uses the dominant 7th chord, which he usually
shuns. Another instructive example is Desolation Row, where drop C is
used on the album, drop D in the live shows of 1965/66.

The three tunings had their periods. \emph{Double drop D} is a thing
of the early days. Since the third in the D chord (on the first
string) is gone, it's a perfect tuning for modal, folky songs like
Ballad of Hollis Brown or John Brown, or blues tunes like Rocks And
Gravel, Motherless Children, West Texas and Quit Your Low Down
Ways. \emph{Drop D} is also favoured in the early days. It is not as
insistently a D-ish tuning as double drop D -- it is more versatile,
used both as a folky, modal tuning as in Barbara Allen or Masters of
War, and as a way of varying the sound of standard three-chord songs
like Mr Tambourine Man. \emph{Drop C} is the favoured tuning in
1965/66, both solo, with Robbie in hotel rooms, and with the band on
stage. It gives a very forceful foundation, thanks to the doubled C in
the bottom. \\ The merit of all these tunings is the fuller sound they
produce. This may be a need felt by a solo acoustic act, but in a
band, there is a bass player to fulfill that function. Double drop D
disappeared very early, and there are no drop C songs after the 1966
tour. But on two songs he has been faithful to drop D, throughout his
carreer: ``It's alright ma'' and ``A Hard Rain's A-Gonna
Fall''.

\emph{Songs in Double Drop D.} Down the Highway --- Ballad of Hollis
Brown --- John Brown --- Rocks And Gravel --- Motherless Children ---
West Texas --- Quit Your Low Down Ways.

\emph{Songs in Drop D.} Gospel Plow --- See That My Grave is Kept
Clean --- Fixin' to Die --- Long Ago, Far Away --- Masters of War ---
A Hard Rain's A-Gonna Fall --- Handsome Molly --- Cuckoo Is A Pretty
Bird --- Barbara Allen --- I ode Out One Morning --- James Alley Blues
--- Mr Tambourine Man --- It's Alright Ma --- Desolation Row --- Tell
me Momma --- House Carpenter --- Broke Down Engine

\emph{Songs in Drop C.} The Two Sisters (1960) --- It's All Over Now,
Baby Blue --- Love Minus Zero/No Limit --- Desolation Row ---
4th Time Around --- Sad-Eyed Lady of the
Lowlands, Absolutely Sweet Marie --- Just Like a
Woman --- I Wanna Be Your Lover --- Farewell Angelina --- On A Rainy
Afternoon/Does She Need Me? --- What Kind Of Friend Is This?


\subsection*{Fingering}

\textbf{The F word}. Uh, chord. You will not get far in the world of
Dylan songs if you can't finger it. In general, what one can say about
the F chord applies to all chords where you need to finger all the
strings. There are four ways:

\emph{Barre chords.} This requires a strong index finger, but, perhaps
even more, a relaxed hand: you should not press too hard either. Your
hand should know (from experience) just how hard you have to press to
make all the strings sound clean, but without straining you hand.

\emph{Use your thumb.} Any classical guitar teacher would kill me for
saying this (and then he would kill \emph{you} for following my
advice), but in a sense, while they shoot me through the head, they
are also shooting themselves in the foot. The reason for the classical guitarists' ``{}thumb
always below the middle of the neck''{} rule is to ensure economy of means: a maximum
flexibility with a minimum of physical effort, but if you don't
need to play chords like 243115, which you don't (it can be done,
though\ldots{}), the most economical thing \emph{is} to use the
thumb. The switch between C=332010 and F=133211 is much, much easier
and smoother with the thumb-F than with the barre-F. Again, you don't
have to push very hard to get the sound you need. (Besides, there is
no way on earth you are ever going to look as cool as Keith Richards
if you only play barre chords.)

\emph{Use only some strings, and/or open strings.} You don't always
have to finger all six strings. If you play with an emphasis on the
bass, you can do with 133xxx, or if you need the full chord or a
brighter sound, xx3211 is perfectly acceptable. In the latter case,
you can even play x03211, since \textbf{a} is part of the F chord. A
similar case is B flat, a terrible chord to finger the ordinary way
(x13331), but much more playable as xx0331 or x5333x.

This is not limited to ``{}standard''{} barre chords; a chord like A
benefits strongly from a barre treatment (with or without the first
string), both because it is easier, and because you can then easily
switch to D/a=x04232. A half barre on the middle strings (A=x02220
with the index finger bent at the last joint) is a handy technique to
have acquired.

\emph{Cheat.} Try some closely related chords instead, like D minor or
A minor, or allow some open strings (x03210, e.g.).


\paragraph*{G major.} This chord should be fingered with the middle, ring,
and little fingers. This leaves the index finger free to do other
things, or to move in position for the C chord which is very likely to
follow, at least in Dylan's idiom. This is particularly true of the
embellishing figure G -- C/g -- G (320003 -- 3x2013 -- 320003) which
you will find all over Dylan's output. Watch Joan Baez do that with
the ``{}index-middle-ring-finger G''{} in \emph{Renaldo \& Clara }(or is it the
\emph{Hard Rain }TV special?), then go and rehearse the ``{}pinky
G''{} instead (I cringe everytime I watch that sequence) (she's cute, though).

And again, cheating can be a good thing. You may not need the first
string: 32000x is perfectly legitimate, and should it happen to sound
anyway (320000), no big harm is done -- you're just playing G6
instead\ldots{}


\paragraph*{Dampening.} Sometimes you have to dampen some strings. To play
G11=3x3211 you need both the sixth string, which is the only
\textbf{g} in there, and all the others, but you don't want the
\textbf{a} on the fifth string. You have to mute it with the ring
finger.

F6 is an even trickier chord. It has to be played 13x231, because you
need both the \textbf{c} and the \textbf{d}. Again, the ring finger
does the muting. (Another alternative is to play xx3535).

\paragraph*{``How on earth\ldots} am I supposed to play 355443 from
`In the Garden'?'' Answer: you're not. You pick some of the strings, perhaps
different strings each time. It's a bit mean of me to write a chord
like that, but my intentions are good.

Incidentally (and you are never going to need this for playing Dylan),
243115, which I mentioned above, can be played with a ``{}twisted barre''{}, with an index
finger that covers both the two 1s (second and third string) and the 2
on the sixth string. The chord can be called
F$\sharp$mmaj9-5. (Exercise 1: find out why. Exercise 2: find other
names for it. [``{}Gerald''{} is not a legitimate answer.])


\subsection*{Fingerpicking}

Although he doesn't use it much these days, many of the old songs use
what I call ``{}standard fingerpicking''{}. I don't know if there is
such a thing, but here is what I mean, as an example. (Chords: G and
C. `{}h'{} in the second measure means hammer-on)

\begin{pre}
\begin{adjustwidth}{2em}{0em}
~~G~~~.~~~.~~~.~~~~~C/g~.~~~.~~~.\\*
|{-}3{-}{-}{-}{-}{-}{-}{-}{-}{-}{-}{-}{-}{-}{-}{-}|{-}{-}{-}{-}{-}{-}{-}{-}{-}{-}{-}3{-}{-}{-}{-}{-}| ring finger\\*
|{-}{-}{-}{-}{-}{-}{-}{-}{-}{-}{-}0{-}{-}{-}{-}{-}|{-}0h1{-}{-}{-}{-}{-}{-}{-}{-}{-}{-}{-}{-}{-}| middle finger\\*
|{-}{-}{-}{-}{-}{-}{-}0{-}{-}{-}{-}{-}{-}{-}0{-}|{-}{-}{-}{-}{-}{-}{-}0{-}{-}{-}{-}{-}{-}{-}0{-}| index finger\\*
|{-}{-}{-}{-}{-}0{-}{-}{-}{-}{-}{-}{-}0{-}{-}{-}|{-}{-}{-}{-}{-}2{-}{-}{-}{-}{-}{-}{-}2{-}{-}{-}| thumb\\*
|{-}{-}{-}{-}{-}{-}{-}{-}{-}{-}{-}{-}{-}{-}{-}{-}{-}|{-}{-}{-}{-}{-}{-}{-}{-}{-}{-}{-}{-}{-}{-}{-}{-}{-}| (thumb)\\*
|{-}3{-}{-}{-}{-}{-}{-}{-}3{-}{-}{-}{-}{-}{-}{-}|{-}3{-}{-}{-}{-}{-}{-}{-}3{-}{-}{-}{-}{-}{-}{-}| thumb
\end{adjustwidth}
\end{pre}

The variations are of course unlimited, but the main principle is as
rock solid as the thumb ought to be: The thumb alternates between the
bass strings, and the other fingers fill in.

For examples of different patterns, more or less fully written out,
see the following songs:

Girl of the North Country (\textit{several versions, fully written
out}) --- Boots of Spanish Leather (\textit{same song, musically
speaking}) --- Percy's Song --- Don't Think Twice, It's All Right
(\textit{basically simple, if it wasn't for all the little
details\ldots}) --- Suze (The Cough Song) --- Cocaine Blues ---
Barbara Allen (\textit{``tricky licks'' department}) --- Rocks and
Gravel --- Seven Curses (\textit{quite similar to \textnormal{Rocks and
Gravel}}) --- Buckets of Rain (\textit{open E tuning}) --- Tomorrow is
a Long Time (\textit{standard and open E tuning})

%\subsection*{Harp Keys}

%Christer Svensson has compiled the following list of harmonica keys to
%Dylan's songs.

\end{articlelayout}
